% Options for packages loaded elsewhere
\PassOptionsToPackage{unicode}{hyperref}
\PassOptionsToPackage{hyphens}{url}
%
\documentclass[
]{article}
\usepackage{amsmath,amssymb}
\usepackage{lmodern}
\usepackage{iftex}
\ifPDFTeX
  \usepackage[T1]{fontenc}
  \usepackage[utf8]{inputenc}
  \usepackage{textcomp} % provide euro and other symbols
\else % if luatex or xetex
  \usepackage{unicode-math}
  \defaultfontfeatures{Scale=MatchLowercase}
  \defaultfontfeatures[\rmfamily]{Ligatures=TeX,Scale=1}
\fi
% Use upquote if available, for straight quotes in verbatim environments
\IfFileExists{upquote.sty}{\usepackage{upquote}}{}
\IfFileExists{microtype.sty}{% use microtype if available
  \usepackage[]{microtype}
  \UseMicrotypeSet[protrusion]{basicmath} % disable protrusion for tt fonts
}{}
\makeatletter
\@ifundefined{KOMAClassName}{% if non-KOMA class
  \IfFileExists{parskip.sty}{%
    \usepackage{parskip}
  }{% else
    \setlength{\parindent}{0pt}
    \setlength{\parskip}{6pt plus 2pt minus 1pt}}
}{% if KOMA class
  \KOMAoptions{parskip=half}}
\makeatother
\usepackage{xcolor}
\IfFileExists{xurl.sty}{\usepackage{xurl}}{} % add URL line breaks if available
\IfFileExists{bookmark.sty}{\usepackage{bookmark}}{\usepackage{hyperref}}
\hypersetup{
  hidelinks,
  pdfcreator={LaTeX via pandoc}}
\urlstyle{same} % disable monospaced font for URLs
\usepackage[margin=1in]{geometry}
\usepackage{graphicx}
\makeatletter
\def\maxwidth{\ifdim\Gin@nat@width>\linewidth\linewidth\else\Gin@nat@width\fi}
\def\maxheight{\ifdim\Gin@nat@height>\textheight\textheight\else\Gin@nat@height\fi}
\makeatother
% Scale images if necessary, so that they will not overflow the page
% margins by default, and it is still possible to overwrite the defaults
% using explicit options in \includegraphics[width, height, ...]{}
\setkeys{Gin}{width=\maxwidth,height=\maxheight,keepaspectratio}
% Set default figure placement to htbp
\makeatletter
\def\fps@figure{htbp}
\makeatother
\setlength{\emergencystretch}{3em} % prevent overfull lines
\providecommand{\tightlist}{%
  \setlength{\itemsep}{0pt}\setlength{\parskip}{0pt}}
\setcounter{secnumdepth}{-\maxdimen} % remove section numbering
\ifLuaTeX
  \usepackage{selnolig}  % disable illegal ligatures
\fi

\author{}
\date{\vspace{-2.5em}}

\begin{document}

\spacing{1}

\hypertarget{review-of-literature}{%
\section{Review of Literature}\label{review-of-literature}}

There have been numerous articles and papers within the scope of using
ML methods to predict credit default, this includes the prediction for
credit card default.\\
\vspace{.3 cm} Studies examining credit card default have been
concentrated, mainly using data originally used as part of
\citet{YEH20092473}. This data is currently freely available as the
Default credit card clients Data Set on the UC Irvine Machine Learning
Repository.\footnote{\url{https://archive.ics.uci.edu/ml/datasets/default+of+credit+card+clients}}
This data, collected in October 2005 is from a cash and credit card
issuing bank in Taiwan. Among the 30,000 observations, 22.12\% are the
cardholders who defaulted on payment. This research employed a binary
variable, default payment (Yes = 1, No = 0), as the response variable.
This data uses 23 explanatory variables, including a mix of personal
information(age, gender, marital status and education level), amount of
credit given, historical bill statements, and historical payment
information.\\
\vspace{.3 cm} \citet{YEH20092473}, compared six classification
algorithms - K-nearest neighbour, Logistic regression, Discriminant
analysis Na\char239 ve Bayesian classifier, Artificial Neural Networks,
Classification Trees. In the classification accuracy, the results show
that there were few differences in error amongst the six methods. The
generated probability of default by the Artificial Neural Network most
closely resembled the actual probability of default which was estimated
using a novel ``Sorting Smoothing Method''.\\
\vspace{.3 cm} Other research on predicting credit card default in
subsequent years utilised this data for training and evaluating models.
The following is a sample of articles available, applying a wide variety
of Machine Learning methods to this classification problem.
\citet{Neema2017TheCO}, took a similar approach in the choice of methods
but attempted to predict the best possible cost-effective outcome from
the risk management perspective. Naive Bayes Estimator and Random Forest
classifiers were introduced. This was evaluated using a cost function
which gave a higher cost to defaulters classified not correctly as they
are the minority in the data. It was concluded that original data with
the Random Forest algorithm is the best in terms of a good balance on
cost versus accuracy.\\
\vspace{.3 cm} \citet{Yang2018-rt} introduces two new methods used to
predict credit card default. Support Vector Machine (SVM) involves using
a kernel function to map the predictor data into a high-dimensional
feature space where the outcome classes are easily separable. XGBoost
and LightGBM, forms of gradient boosted trees algorithm were used.
LightGBM and XGBoost were both deemed to have the best performance in
the prediction of categorical response variables.\\
\vspace{.3 cm} While other studies have used Neural Networks in
predicting credit card defaults, the models used have been vague, and
little detail has been given on the architecture or tuning of the model.
\citet{dnn2} trialled a range of Networks, experimenting with two to
five layers with the number of processing units of 64, 32 and 16 units.
Neural Networks with three layers and 64 units recorded the highest
accuracy of all configurations.\\
\vspace{.3 cm} Due to a lack of credit card-specific data pertaining to
defaulting on payment, all available studies which predict credit card
default utilise the Taiwan data, which is both region specific, over 15
years old and obtained at a time when credit card issuers in Taiwan
faced a credit card debt crisis.\footnote{\url{https://sevenpillarsinstitute.org/case-studies/taiwans-credit-card-crisis/}}

\end{document}
