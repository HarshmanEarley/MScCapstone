% Options for packages loaded elsewhere
\PassOptionsToPackage{unicode}{hyperref}
\PassOptionsToPackage{hyphens}{url}
%
\documentclass[
]{article}
\usepackage{amsmath,amssymb}
\usepackage{lmodern}
\usepackage{iftex}
\ifPDFTeX
  \usepackage[T1]{fontenc}
  \usepackage[utf8]{inputenc}
  \usepackage{textcomp} % provide euro and other symbols
\else % if luatex or xetex
  \usepackage{unicode-math}
  \defaultfontfeatures{Scale=MatchLowercase}
  \defaultfontfeatures[\rmfamily]{Ligatures=TeX,Scale=1}
\fi
% Use upquote if available, for straight quotes in verbatim environments
\IfFileExists{upquote.sty}{\usepackage{upquote}}{}
\IfFileExists{microtype.sty}{% use microtype if available
  \usepackage[]{microtype}
  \UseMicrotypeSet[protrusion]{basicmath} % disable protrusion for tt fonts
}{}
\makeatletter
\@ifundefined{KOMAClassName}{% if non-KOMA class
  \IfFileExists{parskip.sty}{%
    \usepackage{parskip}
  }{% else
    \setlength{\parindent}{0pt}
    \setlength{\parskip}{6pt plus 2pt minus 1pt}}
}{% if KOMA class
  \KOMAoptions{parskip=half}}
\makeatother
\usepackage{xcolor}
\IfFileExists{xurl.sty}{\usepackage{xurl}}{} % add URL line breaks if available
\IfFileExists{bookmark.sty}{\usepackage{bookmark}}{\usepackage{hyperref}}
\hypersetup{
  hidelinks,
  pdfcreator={LaTeX via pandoc}}
\urlstyle{same} % disable monospaced font for URLs
\usepackage[margin=1in]{geometry}
\usepackage{graphicx}
\makeatletter
\def\maxwidth{\ifdim\Gin@nat@width>\linewidth\linewidth\else\Gin@nat@width\fi}
\def\maxheight{\ifdim\Gin@nat@height>\textheight\textheight\else\Gin@nat@height\fi}
\makeatother
% Scale images if necessary, so that they will not overflow the page
% margins by default, and it is still possible to overwrite the defaults
% using explicit options in \includegraphics[width, height, ...]{}
\setkeys{Gin}{width=\maxwidth,height=\maxheight,keepaspectratio}
% Set default figure placement to htbp
\makeatletter
\def\fps@figure{htbp}
\makeatother
\setlength{\emergencystretch}{3em} % prevent overfull lines
\providecommand{\tightlist}{%
  \setlength{\itemsep}{0pt}\setlength{\parskip}{0pt}}
\setcounter{secnumdepth}{-\maxdimen} % remove section numbering
\ifLuaTeX
  \usepackage{selnolig}  % disable illegal ligatures
\fi

\author{}
\date{\vspace{-2.5em}}

\begin{document}

\spacing{1.25}

\hypertarget{review-of-literature}{%
\section{Review of Literature}\label{review-of-literature}}

\hypertarget{taiwan-data}{%
\subsection{Taiwan Data}\label{taiwan-data}}

There have been numerous articles and papers within the scope of using
ML methods to predict credit default, this includes the prediction for
credit card default.\\
\vspace{.5 cm} Studies examining credit card default have been
concentrated, mainly using the data used originally used as part of
\citet{YEH20092473} . This data is currently freely available as the
Default credit card clients Data Set on the UC Irvine Machine Learning
Repository.\footnote{\url{https://archive.ics.uci.edu/ml/datasets/default+of+credit+card+clients}}
This data, collected in October 2005, is from a cash and credit card
issuing bank in Taiwan, the targets were credit card holders of the
bank. Among the total 30,000 observations, 22.12\% are the cardholders
who defaulted on payment. This research employed a binary variable,
default payment (Yes = 1, No = 0), as the response variable. This data
uses 23 variables as explanatory variables including a mix of personal
information(age,gender,marital status and education level), amount of
credit given, historical bill statements, and historical payment
information.\\
This initial study, \citet{YEH20092473}, compared six classification
algorithms - K-nearest neighbour, Logistic regression, Discriminant
analysis Nave Bayesian classifier, Artificial Neural Networks,
Classification Trees. In the classification accuracy, the results show
that there are little differences in error amongst the six methods. The
generated probability of default by the Artificial Neural Network most
closely resembled the actual probability of default. The actual
probability of default was estimated using a novel ``Sorting Smoothing
Method''.\\
\vspace{.5 cm} Other research on predicting credit card default is
subsequent years utilised this data to train and evaluate model. The
following is a sample of articles available, applying a wide variety of
Machine Learning methods to this classification problem.\\
Another study, \citep{Neema2017TheCO}, took a similar approach in choice
of methods but attempted to predict the best possible cost-effective
outcome from the risk management perspective. Again, K-Nearest Neighbour
Logistic Regression, Classification Trees and Discriminant Analysis were
evaluated but also Naive Bayes and Random Forest classifiers were
included. This was evaluated using a cost function which gave a higher
cost to defaulters classified not correctly as they are the minority in
the data. Defaulted payments can prove more costly to a bank rather than
potential customers wrongly identified as a potential default case. It
was concluded that original data with Random Forest algorithm is the
best in terms of a good balance on cost versus accuracy.\\
\citet{Yang2018-rt} introduces two new methods used to predict credit
card default. Support Vector Machine (SVM) involves using a kernel
function to map the predictor data into a high-dimensional feature space
where the outcome classes are easily separable. XGBoost and LightGBM,
forms of gradient boosted trees algorithm were used, as well as
previously tried methods - Logistic Regression and Neural Networks.
LightGBM and XGBoost were both deemed to have the best performance in
the prediction of categorical response variables.\\
While other studies have used Neural Networks in predicting credit card
defaults, the models used have been vague and little detail has been
given on architecture or tuning of the model. \citet{dnn2}, trialled a
range of Networks, experimenting with two to five layers with number of
processing units of 64, 32, 16 units. Neural Networks with three layers
and 64 units recorded the highest accuracy of all configurations.\\
\vspace{.5 cm} Due to a lack of credit card specific data pertaining to
defaulting on payment, all available studies which predict credit card
default utilise this data which is both region specific, dated over 15
years at a time when credit card issuers in Taiwan faced a credit card
debt crisis.

\hypertarget{other-literature}{%
\subsection{Other Literature}\label{other-literature}}

In the scope of predicting defaults on other forms of credit such as
mortgages, research has also been conducted using data collected by the
Central Bank of Ireland, comprising four separate portfolios of over
300,000 owner-occupier mortgage loans of Irish lenders,
\citet{FITZPATRICK2016427}. It was found that boosted regression trees
provided the best classification algorithms for mortgage default
prediction. \vspace{.5 cm}

A 2019 research thesis, \citet{trap5146}, examined a number of
high-performing methods in predicting credit default on a Home Credit
dataset\footnote{\url{https://www.kaggle.com/c/home-credit-default-risk}}.
Home Credit is an international non-bank financial institution that
specializes in lending to people with little or no credit history. This
study examined methods that could be deemed explainable - - which
consisted of a number of tree-based ensemble methods. The top performing
model was deemed to be XGBoost, a form of gradient boosted trees
algorithm.

\end{document}
